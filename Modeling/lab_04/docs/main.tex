\documentclass[14pt]{extreport}
\usepackage[utf8]{inputenc}
\usepackage[english, russian]{babel}
\usepackage{listings}
\usepackage{graphicx}
\usepackage{float}
\graphicspath{{imgs/}}
\usepackage{amsmath,amsfonts,amssymb,amsthm,mathtools} 
\usepackage{pgfplots}
\usepackage{filecontents}
\usepackage{indentfirst}
\usepackage{eucal}
\usepackage{enumitem}
\frenchspacing

\usepackage{indentfirst} % Красная строка

\usetikzlibrary{datavisualization}
\usetikzlibrary{datavisualization.formats.functions}

\usepackage{amsmath}
\usepackage{fixltx2e}
\usepackage{caption}


\definecolor{bluekeywords}{rgb}{0,0,1}
\definecolor{greencomments}{rgb}{0,0.5,0}
\definecolor{redstrings}{rgb}{0.64,0.08,0.08}
\definecolor{xmlcomments}{rgb}{0.5,0.5,0.5}
\definecolor{types}{rgb}{0.17,0.57,0.68}

\usepackage{listings}
\lstset{language=[Sharp]C,
	captionpos=t,
	numbers=left, %Nummerierung
	numberstyle=\small, % kleine Zeilennummern
	frame=single, % Oberhalb und unterhalb des Listings ist eine Linie
	stepnumber=1,                   
	numbersep=5pt,                
	showspaces=false,
	tabsize=2,
	showtabs=false,
	breaklines=true,
	showstringspaces=false,
	breakatwhitespace=true,
	escapeinside={(*@}{@*)},
	commentstyle=\color{greencomments},
	morekeywords={partial, var, value, get, set},
	keywordstyle=\color{bluekeywords},
	stringstyle=\color{redstrings},
	basicstyle=\ttfamily\small,
}

\usepackage[left=1cm,right=1cm, top=1cm,bottom=2cm,bindingoffset=0cm]{geometry}
% Для измененных титулов глав:
\usepackage{titlesec, blindtext, color} % подключаем нужные пакеты
\definecolor{gray75}{gray}{0.75} % определяем цвет
\newcommand{\hsp}{\hspace{20pt}} % длина линии в 20pt
% titleformat определяет стиль
\titleformat{\chapter}[hang]{\Huge\bfseries}{\thechapter\hsp\textcolor{gray75}{|}\hsp}{0pt}{\Huge\bfseries}

\usepackage{array}
\newcommand{\head}[2]{\multicolumn{1}{>{\centering\arraybackslash}p{#1}}{#2}}

% plot
\usepackage{pgfplots}
\usepackage{filecontents}
\usetikzlibrary{datavisualization}
\usetikzlibrary{datavisualization.formats.functions}

\begin{document}
	%\def\chaptername{} % убирает "Глава"
	\thispagestyle{empty}
	\begin{titlepage}
		\noindent \begin{minipage}{0.15\textwidth}
			\includegraphics[width=\linewidth]{b_logo}
		\end{minipage}
		\noindent\begin{minipage}{0.9\textwidth}\centering
			\textbf{Министерство науки и высшего образования Российской Федерации}\\
			\textbf{Федеральное государственное бюджетное образовательное учреждение высшего образования}\\
			\textbf{~~~«Московский государственный технический университет имени Н.Э.~Баумана}\\
			\textbf{(национальный исследовательский университет)»}\\
			\textbf{(МГТУ им. Н.Э.~Баумана)}
		\end{minipage}
		
		\noindent\rule{18cm}{3pt}
		\newline\newline
		\noindent ФАКУЛЬТЕТ $\underline{~~~~~~~~~~~~~~~~~~~~~~~~~~~~~~~\text{«Информатика и системы управления»}~~~~~~~~~~~~~~~~~~~~~~~~~~~~~~~~~~~~~}$ \newline\newline
		\noindent КАФЕДРА $\underline{~~~~~~~~~~~~~\text{«Программное обеспечение ЭВМ и информационные технологии»}~~~~~~~~~~~~~~~~~~~~~~~}$\newline\newline\newline\newline\newline\newline\newline\newline
		
		
		\begin{center}
			\noindent\begin{minipage}{1.3\textwidth}\centering
				\Large\textbf{  Отчет по лабораторной работе №4}\newline
				\textbf{по дисциплине \newline "Моделирование"}\newline\newline
			\end{minipage}
		\end{center}
		
		\noindent\textbf{Тема} $\underline{\text{Моделирование работы системы массового обслуживания}}$\newline\newline
		\noindent\textbf{Студент} $\underline{\text{Малышев И. А.}}$\newline\newline
		\noindent\textbf{Группа} $\underline{\text{ИУ7-71Б}}$\newline\newline
		\noindent\textbf{Оценка (баллы)} $\underline{\text{~~~~~~~~~~~~~~~~~~~~~~~~~~~}}$\newline\newline
		\noindent\textbf{Преподаватель: } $\underline{\text{Рудаков И. В.}}$\newline\newline\newline
		
		\begin{center}
			\vfill
			Москва~---~\the\year
			~г.
		\end{center}
	\end{titlepage}
	
	
	\setcounter{page}{2}

\chapter{Задание}

Смоделировать работу системы, состоящей из генератора, очереди и обслуживающего аппарата. Генерация заявок происходит по закону равномерного распределения с заданными параметрами a, b. Обработка заявок происходит по закону распределения Гаусса с заданными параметрами $\mu$, $\sigma$.

Требуется определить длину очереди, при которой не будет потери сообщений.

Также смоделировать работу системы с построенной обратной связью, в качестве параметра используется процент обработанных заявок, вновь поступивших на обработку.

Протяжка модельного времени должна осуществляться по $\Delta t$ и по событийному принципу. Обозначить, есть ли разница в результатах. 



\chapter{Решение}
\section{Теоретическая часть}

\subsection{Система массового обслуживания}

СМО -- это система, которая производит обслуживание поступающих в неё требований. Обслуживание требований в СМО осуществляется обслуживающими аппаратами. Классическая СМО содержит в себе от одного до бесконечного числа подобных аппаратов.

\subsection{Протяжка модельного времени по $\Delta t$}

Принцип $\Delta t$ заключается в последовательном анализе состояний всех элементов системы в некоторый момент времени $t + \Delta t$ по заданному состоянию этих элементов в момент времени $t$. При этом новое состояние элементов определяется в соответствии с их алгоритмическим описанием с учётом действующих случайных факторов, задаваемых распределениями вероятности. В результате такого анализа принимается решение о том, какие общесистемные события должны имитироваться программной моделью на текущий момент времени.

\subsection{Протяжка модельного времени по событийному принципу}

Характерным свойством систем обработки информации является тот факт, что состояние отдельных элементов изменяется в некоторые дискретные моменты времени, совпадающие с моментами времени поступления сообщений в систему и так далее. Моделирование и продвижение времени в системе посему так же удобно проводить, используя событийный принцип, при котором состояние всех элементов имитационной модели анализируется лишь в момент появления какого-либо события. Момент поступления следующего события определяется минимальным значением из списка будущих событий, представляющего собой совокупность моментов ближайшего изменения состояния каждого из элементов системы.

\section{Листинг}

Далее представлен фрагмент программы, выполняющий поставленное задание.

\begin{lstlisting}

\end{lstlisting}

\section{Результаты работы}

На рисунках \ref{img:ui1}-\ref{img:ui2} представлен пользовательский интерфейс программы в исходном состоянии.

\begin{figure}[H]
	\begin{center}
		\includegraphics[scale=0.8]{imgs/}
	\end{center}
	\caption{Пользовательский интерфейс программы до генерации чисел.}
	\label{img:ui1}
\end{figure}

\begin{figure}[H]
	\begin{center}
		\includegraphics[scale=0.8]{imgs/}
	\end{center}
	\caption{Пользовательский интерфейс программы до ввода чисел.}
	\label{img:ui2}
\end{figure}

\subsection{Пример работы}

На рисунках \ref{img:res} представлен пример результатов работы программы с указанными данными.

\begin{figure}[H]
	\begin{center}
		\includegraphics[scale=0.8]{imgs/}
	\end{center}
	\caption{Исходные данные и результат.}
	\label{img:res}
\end{figure}

\end{document}